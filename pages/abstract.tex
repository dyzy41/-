本文聚焦于高分辨率遥感影像变化检测(RSCD)领域的关键挑战,围绕变化检测基础范式创新、特征提取强化、双时相特征交互、以及伪变化抑制四个核心层面,展开了系统性的研究,提出了一系列系统性的解决方案,旨在全面提升变化检测模型的精度、鲁棒性与泛化能力,为环境监测、城市规划及灾害评估等应用提供先进的变化检测技术支持。

首先,在变化检测基础范式创新方面,为突破传统架构的局限性,本文提出了一个颠覆性的Siamese-Encoder-Exchange-Decoder (SEED)框架。该范式摒弃了现有模型普遍依赖的显式“差异计算”或“特征混合”模块,仅通过在孪生网络中进行特征交换,驱动模型基于“像素一致性原则”来隐式地学习和定位双时相特征的“不一致性”。SEED架构不仅在结构上更为简洁、参数更高效,避免了传统融合操作可能带来的信息损失,更在理论上统一了变化检测与语义分割的框架。实验证明,该范式具备卓越的性能和高度的灵活性,为未来变化检测模型的设计与技术迁移提供了全新的视角。

其次,在特征提取强化方面,为解决传统模型因语义先验不足导致的泛化能力受限问题,本文探索了基于AI基础模型的两条技术路径。一是提出了ChangeCLIP模型,首次将多模态视觉-语言预训练模型(CLIP)引入变化检测。该模型通过利用CLIP的无监督推理能力生成文本提示,并设计了差异特征补偿(DFC)模块与基于低秩双线性注意力的视觉-语言驱动解码器,能够为遥感影像显式注入丰富的语义信息,有效弥补了单一视觉模态在复杂场景理解上的不足。二是提出了PeftCD框架,系统性地研究了如何通过参数高效微调(PEFT)技术(如LoRA、Adapter),将SAM2、DINOv3等强大的视觉基础模型高效地适配于遥感场景。该框架在保留基础模型强大通用先验的同时,以极低的训练成本实现了对下游任务的快速迁移,实验证明其在多个变化检测数据集上均取得了先进的性能。

再者,在双时相影像差异表征方面,本文设计了多种创新的双时相特征处理模块。首先,提出基于欧氏距离度量的EfficientCD,它将双时相特征几何距离显式地融入逐层解码过程,直观地增强了对变化区域的敏感性。同时,结合特征交换策略,提出了特征交换金字塔模块,强化了双时相影像之间的信息交流。其次,提出融合了通道与空间双重差异的Layer-Exchange Network(LENet),将变化特征学习从空间维度拓展到通道维度,构建了通道-空间差异学习模块,促使模型从不同的维度表征变化。这些方法显著提升了模型对复杂变化特征的感知与定位能力。

最后,在伪变化抑制方面,为解决由光照、季节等因素引起的双时相影像“风格”差异导致的域偏移问题,本文提出了内容与风格解缠的解决方案。设计的Content-Style Disentanglement Network (CSDNet),通过内容-风格解耦模块和上下文内容细化模块,在特征层面主动将影像信息分离为代表地物本质的“内容”信息和代表成像条件的“风格”信息。通过实例归一化剥离风格信息,并利用门控机制自适应地过滤无关的风格噪声,使模型能更专注于真实的地物内容变化,从而大幅提升了在复杂成像条件下的鲁棒性。

综上所述,针对于变化检测任务,本文基于基础范式创新、特征提取强化、差异表征以及伪变化抑制四个方面进行了研究,为高分辨率遥感影像变化检测技术的发展注入了新的思路、方法与范式。在多个公开数据集上的详尽实验表明,本文所提出的系列模型在各项关键指标上均取得了领先或极具竞争力的性能,具有重要的理论价值与工程应用前景。
