本文的主要创新在于特征交换策略在变化检测任务中的应用,提出了新型的孪生-编码-交换-解码(\url{Siamese-Encoder-Exchange-Decoder}, SEED)架构,摆脱了变化检测对显式“差异计算”的依赖,通过特征交换机制驱动模型隐式学习特征不一致性,不仅简化
了变化检测范式,还统一了变化检测与语义分割的框架。在以上框架创新基础上,提出如下三个具体创新点:
\begin{enumerate}[label=(\arabic*)]
  \item  特征提取强化 (第四章): 通过引入多模态信息和利用视觉基础模型进行参数高效微调,提出了首个多模态变化检测模型 ChangeCLIP 以及基于视觉基础模型微调的变化检测模型 PeftCD,有效的将 AI 大模型与变化检测任务进行结合,提升了变化检测任务的算法效果。
  \item  双时相特征交互 (第五章): 深入研究了“差异表征”这一核心问题,从数学度量、特征交换以及多维差异等方面提出了促使模型表征差异的解决方案(EfficientCD、LENet),有效的强化了模型对变化的感知能力,提升了变化检测任务的算法效果。
  \item  伪变化抑制 (第六章): 结合 SEED 架构,创新性地提出了内容-风格解缠的思想(Content Style Disentanglement Network,CSDNet),通过 CSDNet 模型,在特征层面主动分离并过滤由光照、季节等引起的风格噪声,显著提升了模型对域差异样本的鲁棒性。
\end{enumerate}